\documentclass[a4paper,12pt]{article}
\title{\textbf{natbib.sty --- versione italiana}\\ 
Prova del formato \emph{plain\_ita}
\\ Vittorio De Martino\\ \small {\texttt{ v.demart@dada.it}}}
\date{Settembre 2002}

\usepackage[italian]{babel}
\usepackage[latin1]{inputenc}
\usepackage{a4}
\usepackage{natbib}

\begin{document}
\maketitle

\begin{center}{\large Attenzione!! \`E consigliabile la compilazione e lettura di test.tex prima di usare questo file}
\end{center}

\section{Introduzione}

\textsf{natbib} � un pacchetto di \LaTeXe\ scritto da P.W.~Daly che consente, utilizzando vari formati (file .bst), di personalizzare la bibliografia  producendo sia  riferimenti nel corpo del documento, sia l'elenco in coda, mediante `numeri' (formato \verb|plain_ita.bst|) o `testo' (formato \verb|natbib_ita.bst|). 

Esistono molte altre opzioni di \textsf{natbib}, per le quali � bene far riferimento al relativo manuale per approfondimenti.

Supponiamo di disporre di un classico documento \texttt{.tex} pronto per la compilazione e di un documento di database per la bibliografia, \texttt{biblio.bib}. Il documento \texttt{.tex} dovrebbe, ad esempio, essere cos� costituito:

\begin{verbatim}
\documentclass{article}
\usepackage{natbib}
\begin{document}

Il condizionamento alle statistiche ancillari �, alle volte, 
davvero auspicabile \citep{art1}; vedi anche \citep{art2}. 
Comunque non c'� unanimit� sull'argomento \citep[ad es.,][]{art3}.
Altri riferimenti sono \citep{art4} o \citep{art5} o 
\citep{art6} o \citep{art7} ed infine \citep[pg12--20]{art8}.

\bibliographystyle{plain}
\bibliography{biblio}
\end{document}
\end{verbatim}

Il file \texttt{.bib} dovrebbe essere costituito dalla lista dei riferimenti specificandone il tipo (article, PhD thesis, \ldots), ovvero:
\begin{verbatim}
@article{art1,
	author={F. Gargano and G. Lovison },
	title={Some notes on conditional inference},
	journal={J.A.S.A.},
	volume={2},
	year={1999},
	pages={23-45}
}

@phdthesis{art2,
	author={M.T. Morana},
	title={La valutazione di Gargano},
	school={Universit{\`a} di Palermo},
	year={2001}
}
...
\end{verbatim}
 
Compilando con \verb|latex| quindi \verb|bibtex| e poi nuovamente \verb|latex| (due volte) si ottiene il classico documento di output con i riferimenti bibliografici `testuali', ovvero 

``\ldots �, alle volte, davvero auspicabile (1); vedi anche (2) \ldots.''

L'elenco in coda al documento � opportunamente formattato con la lista disposta in ordine alfabetico, ad esempio:\\

\noindent [1] Gargano, F. and Lovison, G. (1999). Some notes on conditional inference. {\sl J.A.S.A.}, \textbf{2}, 23--45\\

\noindent [2] Morana, M. (2001). {\it La valutazione di Gargano.} Ph.D. thesis, Universit� di Palermo.\\ 


\section{La traduzione italiana}
Per l'utilizzo di \textsf{plain} in documenti italiani, le formattazioni di termini anglo-sassoni quali `and', `Ph.D. thesis' non sono adeguati; anche, la virgola dopo il cognome (Morana, M.) dovrebbe essere eliminata. A tal fine � stato fornito un file \verb|natbib_ita.bst| che, non � altro che una modifica del file \texttt{natbib.bst}. \`E sufficiente, a questo punto, utilizzare il comando \verb|\bibliographystyle{natbib_ita}| che chiama il file \verb|natbib_ita.bst| invece che \texttt{natbib.bst}. L'ouput �, in tal caso:\\

``Il condizionamento alle statistiche ancillari �, alle volte, davvero auspicabile \citep{art1}; vedi anche \citep{art2}. Comunque non c'� unanimit� sull'argomento \citep[ad es.,][]{art3}.
Altri riferimenti sono \citep{art4} o \citep{art5} o \citep{art6} o \citep{art7} ed infine \citep[pg12--20]{art8}.''\\

L' elenco in coda al documento risulta evidentemente:

\nocite{*}
%
\bibliographystyle{plain_ita} 
 
\bibliography{biblio}

\end{document}


